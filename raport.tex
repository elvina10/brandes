\documentclass[12pt]{article}
\usepackage {xcolor}
\usepackage {lineno}
\usepackage {graphicx}
\usepackage[T1]{fontenc}
\usepackage[polish]{babel}
\usepackage[utf8]{inputenc}
\usepackage[margin=0.7in]{geometry}
 \setlength{\parindent}{0pt}
 \setlength{\parskip}{0.3cm}

\usepackage{lmodern}
\usepackage{amsfonts}
\usepackage{tikz}
\usepackage{xinttools}
\selectlanguage{polish}
\title {Zadania Domowe Seria I}
\author {Katarzyna Kowalska}
\date{25.11.2015}

\usepackage{amsmath}
%commands matematyka
%logika
%\newcommand {\iff} {\Leftrightarrow}
\newcommand {\wtedy} {\Rightarrow}
\newcommand {\oraz}{\quad\wedge\quad}
\newcommand{\Lim}{\raisebox{0.5ex}{\scalebox{0.8}{$\displaystyle \lim_{n \rightarrow \infty}\;$}}}

%macierze
\DeclareMathOperator{\im}{im}
\DeclareMathOperator{\rank}{rank}

\usepackage{fancyhdr}
\pagestyle{fancy}

\lhead{Opis algorytmu Brandesa}
\rhead{Katarzyna Kowalska}

\begin{document}
	\section* {Opis rozwiązania}
		W swoim rozwiązaniu wykorzystuję współbieżność do liczenia algorytmu Brandesa
		zauważając, że liczenie delty dla różnych źródeł jest kompletnie niezależne.
		Każdy wątek otrzyma swój zbiór wierzchołków, które ma przerobić.
		
		Rozdzielam dokładnie tę pętlę, która jest w opisie algorytmu ze strony zaznaczona jako ,,in parallel''.
		Można ją podzielić równo między wszystkie wątki (każdy dostałby kolejne $\frac{n}{d}$ wierzchołków),
		jednakże może się to okazać wysoce nieefektywne, gdyż długość kroku pętli dla danego wierzchołka jako żródła
		może przejść pesymistycznie wszystkie krawędzie w grafie albo odwiedzić tylko źródło (dla wierzchołka o stopniu wychodzącym 0).
		Dlatego każdy wątek, gdy nie ma nic do zrobienia ,,prosi'' o nowy wierzchołek do przerobienia.
		Współbieżne ,,proszenie'' o kolejny wierzchołek kontroluje wzajemne wykluczanie sekcji krytycznej realizowane za pomocą mutexu.

		Ponadto nie deklaruje wielokrotnie tych samych wektorów dla jednego wątku oraz nie czyszczę całej tablicy przy każdym obrocie, a jedynie te pola, których użyłam w tym obrocie pętli oraz wyniki z wielu wątków sumuję na koniec zamiast gwarantować współbieżny dostęp do tablicy BC.
		
	\section* {Wydajność}
		Testy były przeprowadzane na students, podaję średnie czasy z 30 wykonań oraz minimalny i maksymalny czas wykonania. Przyspieszenie jest liczone dla średnich czasów.
		
		\begin{tabular}{lllll}
			Students: & & \\
			Wątki & MIN & MAX & AVG & Przyspieszenie\\
			1 & 7.274 & - \\
			\xintForpair #1#2#3 in {(2,4.098,0),(3,2.411,0),(4,1.990,0),(5,1.733,0), (6, 1.593,0), (7, 1.359,0), (8, 1.261,0)}
			\do
				{#1 & #2 & #3 & & \pgfmathparse{7.274 / #2}\pgfmathresult \\}
		\end{tabular}

\end{document}
